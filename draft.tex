\documentclass[11pt, oneside]{article}
\usepackage[margin=1in]{geometry}
\usepackage{changepage}
\usepackage{graphicx}
\usepackage{amssymb}
\usepackage{amsmath}
\usepackage{nicefrac}
\usepackage{graphicx}
  \graphicspath{ {images/} }
\usepackage[backend=biber,
  style=apa,
  doi=false, isbn=false, eprint=false,
  maxbibnames=10,
  maxcitenames=3,
  language=american]{biblatex}
\DeclareLanguageMapping{american}{american-apa}
\addbibresource{SOC901.bib}
\AtEveryBibitem{\clearfield{month}}
\AtEveryBibitem{\clearfield{day}}

\title{An agent-based model of local food systems}
\author{Yuji Saikai and Elan Segarra}
% \date{}

\begin{document}
\maketitle
\thispagestyle{empty} % remove the page number on the title page
\vspace{5mm}

\begin{abstract}
\noindent 
It seems that local food systems are driven by the factors that play minor roles in conventional food systems, which are characterized by interaction and individual heterogeneity. There are two key interactions: networking \& recommendation among consumers, and trust building between consumers and producers. Diverse motives for food transaction include social, cultural, political, and ethical considerations. In order to understand local food systems, given the dynamical process and heterogeneity, agent-based computational modeling is considered more suitable than the traditional analytical approach.\\\\
\noindent
Keywords: local food system, ABM, trust, network, price competition
\end{abstract}

\pagebreak

\tableofcontents
\pagebreak

\section{Introduction}
Despite the fact that local food systems constitute a small portion of the entire economy, they have been attracting  attention of the general public, policy makers and academics. Interestingly, the reasons for the growing popularity seem to well correspond to the reasons why local food systems occupy a marginalized position in generating income -- ``inefficiency''. There are copious factors contributing to the inefficiency: small-scale production, diverse motives for activities, and interpersonal relationships. If these factors matter to an increasing number of people, however, it may be signaling the existence of alternative social objectives, which call for alternative efficiency metrics and analytical tools. In this paper, we assume such objectives and focus on the development of a more suitable tool to help understand the complex systems where heterogeneous agents dynamically interact for various reasons.

To that end, we appreciate more realism than social scientists normally do. Following \textcite{Miller2009}, we attempt to model realistic ``in-between'' situations. For instance, agents are neither perfectly rational nor completely ignorant; the number of agents are neither a few nor infinite; and space and time are neither absent nor unbounded. Clearly, this is not how traditional analytical models are constructed, and cause and effect is less apparent than in closed-form equations. In order to make the internal mechanics transparent as well as flexible and extendable, we write codes in Python building upon Mesa \parencite{Mesa2016}, an open-source ABM framework. Our codes are also open-source and available on GitHub.com \parencite{Saikai2016}.

The paper first illustrates local food systems and agent-based modeling, neither of which is well recognized in economics. The model is designed to help assess the performance of a given system under alternative mechanisms of interactions and varying configurations of parameters. It is in essence a computational counterpart of comparative statics \parencite{Judd2006}. Therefore, the model description starts with defining the performance metrics. Then, we explain key variables and parameters, followed by details of the underlying interactions. Finally, we briefly provide what is expected as outputs and discuss what to be done.

\section{Local food systems}
(to be written)


\section{Agent-based modeling}
(to be written)


\section{The model}
\subsection{The performance metrics}
Assuming alternative social objectives mentioned above, an ideal performance indicator may be a social welfare function that reflects and aggregates all the diverse values involved in local food systems. This is a very important goal, but should be a far bigger project and clearly beyond the scope of the paper. Instead, we simply adopt as a metrics the number of local producers at a steady state. First, although we do not a priori assume any kind of equilibrium, society usually exhibits some stability with ongoing minor fluctuations. This is what we mean by a steady state and try to generate in the model. Next, to justify the choice of the performance metrics, we pay attention to authenticity of local food. In literature, authenticity is often seen as a key concept to understand because it defines local food and dictates the prosperity of the systems \parencite{Sims2009, Wittman2012}. Combined with the common consumer perception of local food, we regard small-scale producers as ``authentic'' and adopt their numbers as a performance metrics.

\subsection{Agents}
There are two types of agents: consumer $i \in \{1,...,I\}$ and producer $j \in \{1,...,J\}$. At each period, a typical consumer chooses a single producer by comparing the utility obtained upon purchasing food from each producer. The $i$th consumer's utility function of the purchase from the $j$th producer is defined as follows:
\begin{equation} \label{util}
  u_{ij} = \alpha_i t_{ij} + \beta_i e_j - \gamma_i d_{ij} - p_j .
\end{equation}
Through adjusting the price, each producer ``tries to'' maximize the profit:
\begin{equation}  \label{profit}
  \pi_j = p_j Q_j - c_j - k e_j .
\end{equation}
We stress that the producer's maximization is only an attempt due to the bounded rationality. Vendors at a farmers' market, for example, only incorporate neighboring competitors' prices or a limited amount of marketing intelligence. In the next section, we explain the variables and parameters.

\subsection{Variables \& Parameters}
Given the diversity and heterogeneity, the model involves a number of variables and parameters. The following is the only key ones. In addition, since the simulation runs for many periods, each of them may be indexed by time.

\subsubsection{Trust $(t)$}
We use trust as a variable encapsulating all the relevant information about an idiosyncratic producer as an individual, on which a consumer bases an economic decision. It may represent such measures as good \& service quality, morality of business practice, and personal character. The notation $t_{ij}$ indicates the trust level of the $i$th consumer in the $j$th producer.

\subsubsection{Social embeddedness $(e)$}
According to \textcite{Hinrichs2000}, the presence of social embeddedness is often seen as the hallmark of direct agricultural markets. At the same time, she also emphasizes an tension between embeddedness and marketness in each local food enterprise. Therefore, we introduce $e_j \in [0,1]$ capturing the embeddedness of the $j$th producer's enterprise. We interpret embeddedness as an objective property that reflects the socialness of the enterprise. In this way, it is distinguished from trust build upon the producer's personal characteristic. As a result, it enters into both objective functions, positive in utility and negative in profit, reflecting respectively consumer preference toward social experience in economic transaction and producers' customer maintenance costs.

\subsubsection{Distance $(d)$}
Figure 1 shows the grid space of the model. Over 400 cells, there are randomly allocated 49 local producers (green dots), 1 conventional producer (red dot), and 300 consumers (suppressed). $d_{ij}$ is equal to the sum of x-coordinate difference and y-coordinate difference between the $i$th consumer and the $j$th producer.
\begin{figure}[h]
\centering
  \includegraphics[scale=0.8]{img01}
  \caption{The grid space}
\end{figure}

\subsubsection{Others $(p, c, \alpha, \beta, \gamma, k)$}
As seen in equation \ref{util} \& \ref{profit}, $p$ is a unit price, and the revenue for the $j$th producer is equal to $p_j$ times $Q_j$ (the total sales as well as the number of customers). $c$ is the fixed costs. Each of the parameters $c, \alpha, \beta, \gamma, k$ straightforwardly captures a marginal effect of the corresponding variable.

For the sake of analytical simplicity, we adopt the linear functional forms. However, this necessarily makes the interpretation of each parameter problematic. For instance, since 1 dollar is set equal to 1 util through $p$ and $c$, 1 trust level must be equivalent to $\alpha$ dollars. We do not yet have good interpretations of these coefficients, and so it is of critical importance to figure out appropriate functional forms.

\subsection{Interactions}
At each period, all the consumers move first, followed by all the producers. The order within a type of agents is random, which is immaterial under the current specification. 

\subsubsection{Purchase}
In maximizing utility, besides individual information (i.e. trust and distance), consumers are assumed to have access to each producer's price $(p_j)$ and embeddedness level $(e_j)$. At each period, consumers create a list of possible utilities and then, based on the relative utilities as probability weights, randomly choose a single producer to buy from. The reason for the stochastic process is the fact that consumers do not always strictly follow their internal ranking, i.e. random shocks may influence the decisions such as being sick and feeling lazy. This is a matter of data fitting \& calibration of the model. Loosely said, we currently maintain ``moderate chances'' for deviation. Based on \textcite{Walter2008}, we compute the $i$th consumer's weight on $j$th producer $(w_{ij})$ as follows:
\begin{align*}
%  \underline{u}_i &\equiv \min_j\{u_{ij}\} \\
  \Delta_i &\equiv \max_j\{u_{ij}\} - \min_j\{u_{ij}\} \\
  u^*_{ij} &\equiv \exp \left[\frac{15(u_{ij} - \min\{u_{ij}\})}{\Delta_i} - 5 \right] \\
  w_{ij} &= \frac{u^*_{ij}}{\sum_j u^*_{ij}} .
\end{align*}
This way, consumers buy from the highest-ranked producer approximately 60 -- 90\% of the time, and 2/3 of the lowest producers have essentially no chances. Again, it is a matter of calibration and so, if data suggests different choice behavior, we should adjust the parameters and/or the functional forms.

\subsubsection{Trust update}
At each period, a consumer updates her trust levels in all the producers. Here, as well, we try to incorporate social embeddedness into the process. The idea is that the more embedded, the greater chance to interact with the producer and therefore increase the trust. As a result, we assume that an upward adjustment takes place with probability of $0.5 + \nicefrac{e_j}{2}$. Specifically, the $i$th consumer's trust in $j$th producer is updated as follows:
\begin{displaymath}
t'_{ij} =
\begin{cases}
t_{ij} \times 1.05 & \text{with probability } 0.5 + \dfrac{e_j}{2}\\
t_{ij} \times 0.99 & \text{otherwise}
\end{cases}
\end{displaymath}
The mild downward adjustment reflects gradual ``forgetting''. We implicitly set it unbounded from above and equal to 1 from below. The lower bound provides a limit of forgetting and an initial value for new producers. We are aware that both bounds should make sense relative to the other variables in equation \ref{util}. Finally, there is no trust update for conventional producers.

\subsubsection{Pricing}
First of all, we assume that each producer utilizes only local prices in the neighborhood when making an adjustment, whereas consumers consider all the prices upon purchasing decisions. It may seem inconsistent that consumers have access to global price information, while producers take into account only local prices. For consumers, however, a price means literally and only a unit cost of food. In contrast, each producer takes a price as, in addition to a revenue of a unit sale, a strategic variable in competition with other neighboring producers. So, even though producers may know, just like consumers know, all the prices (e.g. learning through the Internet), given the nature of the competition taking place within a limited geographical space, only local prices matter to them.

Given this context, at the beginning of every period, each producer has a random chance to adjust their prices up or down depending on their profits from the previous period. The inclusion of a stochastic element is meant to model the infrequency of price adjustments due to either adjustment costs or simply neglect. The chances of an adjustment occurring are proportional to their situations. For those producers with negative profits, the probability of adjustment is proportional to how far they missed their breakeven point (e.g. a producer with costs of \$50 and revenue of \$20 has a $\nicefrac{30}{50}=60\%$ chance of adjustment). On the other hand, producers with positive profits have chances proportional to how much they missed their ideal profit margin (e.g. a producer with ideal profits of \$20 and with actual profits of \$12 has a $\nicefrac{8}{20}=40\%$ chance of adjustment). Notice that producers with profits above their ideal level will never raise their prices further, even if their sales are extremely high and they are functioning as a monopoly. This process tacitly acknowledges that producers in local food systems tend to strive for a minimum level of profits as opposed to unfettered profit maximization.

As mentioned above, producers do not naively adjust their prices up or down, but instead take into account their environment just as a seller at a farmers' market stall responds to the prices of nearby vendors. If producers find themselves with negative profits and decide to adjust their prices downward, they look in their local neighborhood and undercut the lowest price in an attempt to steal sales and boost their revenue. If they are already the lowest, they keep the prices fixed (implicitly acknowledging that lower prices will only hurt them further). Positive profit producers act similarly; if they are adjusting prices, they raise their prices to undercut their neighbors assuming they find neighbors with higher prices. Otherwise they keep their prices fixed and continue. It is important to notice that price is the only choice variable for producers, and thus is their only means of responding to poor sales and negative profits. Therefore lone producers with negative profits will inevitably close, perhaps an unrealistic result. Future work may include additional response variables, such as adjusting quality or embeddedness, to better model a producer's ability to deal with losses.

\subsubsection{Entry \& Exit}
Since the number of producers is our metric of focus it is natural to include mechanisms for both entry and exit. The exit mechanism is simple and stark: at the beginning of each period any firm with a negative cash balance closes. In future models we may distinguish between shutdown and exit (currently a meaningless distinction since our model lacks variable costs) but for now this mechanism will suffice.

The entry process can be modeled in a myriad of manners which is why that mechanism has gone through several iterations. At one extreme potential entrants have perfect foresight and market information and are able to accurately predict profits at every cell and thus open up shop at the prime profit maximizing location assuming positive profits exist. At the other extreme naive potential entrants only see average profits (or cash balances or sales) and enter the market at a random location if any of these aggregate variables is high enough. To balance computational efficiency and realism we opt for a middle ground. Every period potential entrants look for producers with cash balances that exceed a set level (the entry threshold). If such a producer is found then a new producer enters a cell nearby provided no producers have opened in the neighborhood recently (this prevents any ``pile-up'' effects). Even within this simple mechanism the calibration of the entry threshold and frequency is of prime importance. Our current parameter calibrations result in subjectively reasonable entry and exit fluctuations, however this will need continual adjustment as other elements of our virtual economy are added or subtracted.

\subsubsection{Networking \& Recommendation}
Given the socially embedded nature, connecting consumers and producers is often one of the main objectives of local food systems \parencite{Martinez2010}. The modeling questions are: What is the network for? and How is it formed? It seems true that social interaction in itself is an end, and therefore we may add a connectivity measure to the utility function. Here we see the connections between consumers as a friendship network, over which information about producers is exchanged. As explained above, the consumer's trust in a producer increases only when she makes a purchase, yet trust level is a key determinant of the consumer's decision. Realistically, however, consumers learn about new producers and try them out every now and then. One source of such information is those who participate in the same farmers' market or CSA. In essence, food acts as a medium for interaction, and a seller facilitates network formation among the customers. (n.b. In this section, by `new' producers, we mean those in which a consumer has the lowest trust level.)
%(We regard a collection of individual producers as an institution called a seller.)
\begin{figure}[h]
\centering
  \includegraphics[scale=0.9]{img02}
  \caption{Temporary seller initiated network}
\end{figure}

To implement the idea, first we use an $I \times I$ adjacency matrix \parencite{Jackson2008} to represent the network, which is assumed to be undirected and have zero for all the initial entry. Note that a seller is absent in the permanent network as its role is only to temporarily facilitate interaction among customers at each period. Second, we assume that upon each purchase a consumer has a random chance to communicate and form a tie with another customer in the same seller initiated network. It is random because people do not communicate strictly every single time. The probability is equal to the producer's embeddedness level due to the natural correspondence between embeddedness level and chance for social interaction. Thus, if a consumer buys from a more embedded producer, she has a greater chance for connection, but it does not necessarily mean a new connection as it may be with an existing friend. Once a tie is formed, there is no decay over the subsequent periods. Finally, to specify the mechanism of query and recommendation about a new producer, we have the following story:
\begin{adjustwidth}{1cm}{}
\textit{This week, as usual, after examining all the sellers, she chooses the regular one (trusted and close yet a bit pricy). But, she wonders how a producer sitting at the third place in her ranking might be. She has never tried his food but the price looks good...}
\end{adjustwidth}
So, each consumer sends a query about a new producer, who gives the highest utility among all the new producers. Friends, if any, report back their own trust levels in that producer. She aggregates the information and adopts it as her own trust in the new producer. Specifically, she adopts 90\% of the median of the friends' trust levels.
%Create opportunities for new producers and interesting dynamics

\section{Discussion}
\subsection{Outputs}
Figure 3 shows a typical output of the model, where the graph traces the number of local producers over 156 periods. Though far from complete, we try to define the units of the variables and calibrate the parameters on the basis of 1 period = 1 week. Hence, 156 periods are meant to be 3 years.
\begin{figure}[h]
\centering
  \includegraphics[scale=0.4]{img03}
  \caption{A typical output}
\end{figure}
At this point, there is not much to say about particular outputs, but the underlying mechanics of the model could work. The philosophy of ABM is summarized by what \textcite{Epstein1996} call ``generative'' social science. Before drawing any conclusion, therefore, we must look at data, figure out empirical regularities, apply alternative mechanism of interactions, and calibrate parameters in order to generate as much the empirical phenomena as possible. In what follows, we present a long list of work ahead.

\subsection{Issues}



\vspace{5mm}

\printbibliography[heading=bibintoc]

\end{document}




